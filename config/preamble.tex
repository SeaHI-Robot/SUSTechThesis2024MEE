% !Mode:: "TeX:UTF-8"
% !TEX program  = xelatex

% 数学符号与环境
\usepackage{amsmath,amssymb}
  \newcommand{\dd}{\mathrm{d}}
  \newcommand{\RR}{\mathbb{R}}
  \numberwithin{figure}{section}  % 图片索引按章节编号
% 参考文献
\usepackage[style=gb7714-2015,gbpunctin=false]{biblatex}
  \addbibresource{ref.bib}
% 公式按照章节标号
\usepackage{amsmath}
\numberwithin{equation}{section}
% caption居中
\usepackage[justification=centering]{caption}
% 页眉
\usepackage{fancyhdr}

\fancypagestyle{mystyle}{
    \fancyhf{}
    \fancyhead[C]{\宋体\小五 论文标题(修改config/preamble.tex 21行)} % 页眉添加文章题目
    \renewcommand{\headrulewidth}{0.2pt} % 分隔线宽度0.2磅
    \fancyfoot[C]{\thepage} % 在页脚中间添加页码
}

% \fancyhead[C]{\宋体\小五 点接触式双足机器人的动态行走算法研究}
% \renewcommand{\headrulewidth}{0.4pt}%分隔线宽度4磅
% 无意义文本
\usepackage{zhlipsum,lipsum}
% 列表环境设置
\usepackage{enumitem}
% 浮动题不越过 \section
\usepackage[section]{placeins}
% 超链接
\usepackage{hyperref}
% 图片,子图,浮动题设置
\usepackage{graphicx,subcaption,float}
% 抄录环境设置,更多有趣例子请命令行输入 `texdoc tcolorbox`
\usepackage{tcolorbox}
  \tcbuselibrary{xparse}
  \DeclareTotalTCBox{\verbbox}{ O{green} v !O{} }%
    {fontupper=\ttfamily,nobeforeafter,tcbox raise base,%
    arc=0pt,outer arc=0pt,top=0pt,bottom=0pt,left=0mm,%
    right=0mm,leftrule=0pt,rightrule=0pt,toprule=0.3mm,%
    bottomrule=0.3mm,boxsep=0.5mm,bottomrule=0.3mm,boxsep=0.5mm,%
    colback=#1!10!white,colframe=#1!50!black,#3}{#2}%
\tcbuselibrary{listings,breakable}
  \newtcbinputlisting{\Python}[2]{
    listing options={language=Python,numbers=left,numberstyle=\tiny,
      breaklines,commentstyle=\color{white!50!black}\textit},
    title=\texttt{#1},listing only,breakable,
    left=6mm,right=6mm,top=2mm,bottom=2mm,listing file={#2}}
% 三线表支持
\usepackage{booktabs}

% LaTeX logo
\usepackage{hologo}
